\documentclass[12pt, bibliography=totoc]{scrartcl}
\usepackage[headsepline,automark]{scrlayer-scrpage} %Trennlinie an Kopfzeile
%\usepackage{scrheadings}
\clearpairofpagestyles
\lohead{\rightmark}
\renewcommand{\partmark}[1]{\relax}% \part daran hindern, den Kolumnentitel zu löschen
\ohead[]{\pagemark}
%\ofoot*{\pagemark}
%%Kopfzeile
\usepackage{txfonts} %für times new roman
%\usepackage{helvet} %für arial, dann aber 11pt
\usepackage[a4paper, left=2.5cm, right=5cm]{geometry}
\usepackage[onehalfspacing]{setspace}
\usepackage{wasysym}
\usepackage{rotating}
%\usepackage{caption}
\usepackage[T1]{fontenc}
\usepackage[utf8]{inputenc}
%\usepackage{todonotes}
\usepackage{enumitem}
%\uespackage{caption}
\usepackage[bf]{caption}
\renewcommand{\captionfont}{\small\slshape}
\renewcommand{\figurename}{Abb.}
%\renewcommand{\thefigure}{\arabic{section}.\arabic{figure}}
%\makeatletter \@addtoreset{figure}{section} \makeatother
%\captionsetup[figure]{skip=1pt}
\usepackage{tabularx}
\usepackage{pdfpages}
\usepackage{array}
\usepackage{hyperref}
\usepackage{threeparttable} %fußnoten unterhalb tabelle
\usepackage{booktabs} % fuer schone Tabellen
\usepackage{rotating} % um tabellen auf quer drehen zu koennen http://www.golatex.de/kann-man-tabellen-im-querformat-darstellen-t2003.html
%\newcolumntype{C}[1]{>{\centering\arraybackslash}p{#1}} %Spalten mit fester breite zentriert
\newcolumntype{L}[1]{>{\raggedright\arraybackslash}p{#1}} %Spalten mit fester breite linksbündig
\newcolumntype{Y}{>{\small\raggedright\arraybackslash}X}
\newcolumntype{C}{>{\small\centering\arraybackslash}X}
\usepackage{graphicx}
%\usepackage[german]{babel}
\usepackage{typearea}
%\usepackage[style=apa]{biblatex}
\usepackage[style=apa,backend=biber]{biblatex}
\usepackage[american,ngerman]{babel}
\DeclareLanguageMapping{ngerman}{ngerman-apa}
\usepackage[babel,german=guillemets]{csquotes}

%nach part fängt section wieder mit eins an alte Gestaltung
%\makeatletter
%\@addtoreset{section}{part}
%\makeatother
\renewcommand*{\partformat}{\thepart}{}
\renewcommand*{\partheadmidvskip}{\nobreak\enskip}
%\bibliography{/Users/iNge/Dropbox/Biblio/ingesbibneu}
\bibliography{/Users/iNge/Dropbox/Biblio/library}
%\bibliography{library}
\begin{document}
% das Papierformat zuerst
%\documentclass[a4paper, 11pt]{article}

% deutsche Silbentrennung
%\usepackage[ngerman]{babel}
%\usepackage{color} \color{blue}
% wegen deutschen Umlauten
%\usepackage[utf8]{inputenc}

% hier beginnt das Dokument
%\begin{document}

\begin{titlepage}
\thispagestyle{empty}
\begin{center}
\color{blue}\Large{Fernuniversität Hagen}\\
\end{center}


\begin{center}
%\Large{Bildung und Medien: eEducation}
\end{center}
\begin{verbatim}



\end{verbatim}
\begin{center}
\textbf{\Large{Kommentierte Bibliographie zum Thema Online-Lernen}}
\end{center}
\begin{verbatim}

\end{verbatim}
\begin{center}
%\textbf{im Studiengang Wirtschaftsinformatik}
\end{center}
\begin{verbatim}











\end{verbatim}

\begin{flushleft}
\begin{tabular}{lll}
\textbf{Studiengang:} & & MA Bildung und Medien: eEducation\\
& & Modul 2: Anwendungsbezogene Bildungsforschung\\
& & \\
& & \\
\textbf{eingereicht von:} & & {\color{magenta} Inge Koch-Meinass \flq{}ingekoch@mac.com\frq{}}\\
& & {\color{magenta}Matrikelnr.: 9650962 }\\
& & \\
\textbf{eingereicht am:} & & 09. November 2015\\
& & \\
& & \\
%\textbf{Betreuer:} & & Herr Prof. Dr. J. A. Müller
\end{tabular}
\end{flushleft}

% das ist wohl jetzt das Ende des Dokumentes
\end{titlepage}

%\listoftodos
\tableofcontents
\listoftables
\setcounter{page}{1}
%\thispagestyle{empty}
\pagebreak
