\documentclass[12pt, bibliography=totoc]{scrartcl}
\usepackage[headsepline,automark]{scrlayer-scrpage} %Trennlinie an Kopfzeile
%\usepackage{scrheadings}
\clearpairofpagestyles
\lohead{\rightmark}
\renewcommand{\partmark}[1]{\relax}% \part daran hindern, den Kolumnentitel zu löschen
\ohead[]{\pagemark}
%\ofoot*{\pagemark}
%%Kopfzeile
\usepackage{txfonts} %für times new roman
%\usepackage{helvet} %für arial, dann aber 11pt
\usepackage[a4paper, left=2.5cm, right=5cm]{geometry}
\usepackage[onehalfspacing]{setspace}
\usepackage{wasysym}
\usepackage{rotating}
%\usepackage{caption}
\usepackage[T1]{fontenc}
\usepackage[utf8]{inputenc}
%\usepackage{todonotes}
\usepackage{enumitem}
%\uespackage{caption}
\usepackage[bf]{caption}
\renewcommand{\captionfont}{\small\slshape}
\renewcommand{\figurename}{Abb.}
%\renewcommand{\thefigure}{\arabic{section}.\arabic{figure}}
%\makeatletter \@addtoreset{figure}{section} \makeatother
%\captionsetup[figure]{skip=1pt}
\usepackage{tabularx}
\usepackage{pdfpages}
\usepackage{array}
\usepackage{hyperref}
\usepackage{threeparttable} %fußnoten unterhalb tabelle
\usepackage{booktabs} % fuer schone Tabellen
\usepackage{rotating} % um tabellen auf quer drehen zu koennen http://www.golatex.de/kann-man-tabellen-im-querformat-darstellen-t2003.html
%\newcolumntype{C}[1]{>{\centering\arraybackslash}p{#1}} %Spalten mit fester breite zentriert
\newcolumntype{L}[1]{>{\raggedright\arraybackslash}p{#1}} %Spalten mit fester breite linksbündig
\newcolumntype{Y}{>{\small\raggedright\arraybackslash}X}
\newcolumntype{C}{>{\small\centering\arraybackslash}X}
\usepackage{graphicx}
%\usepackage[german]{babel}
\usepackage{typearea}
%\usepackage[style=apa]{biblatex}
\usepackage[style=apa,backend=biber]{biblatex}
\usepackage[american,ngerman]{babel}
\DeclareLanguageMapping{ngerman}{ngerman-apa}
\usepackage[babel,german=guillemets]{csquotes}

%nach part fängt section wieder mit eins an alte Gestaltung
%\makeatletter
%\@addtoreset{section}{part}
%\makeatother
\renewcommand*{\partformat}{\thepart}{}
\renewcommand*{\partheadmidvskip}{\nobreak\enskip}
%\bibliography{/Users/iNge/Dropbox/Biblio/ingesbibneu}
\bibliography{/Users/iNge/Dropbox/Biblio/library}
%\bibliography{library}
\begin{document}
% das Papierformat zuerst
%\documentclass[a4paper, 11pt]{article}

% deutsche Silbentrennung
%\usepackage[ngerman]{babel}
%\usepackage{color} \color{blue}
% wegen deutschen Umlauten
%\usepackage[utf8]{inputenc}

% hier beginnt das Dokument
%\begin{document}

\begin{titlepage}
\thispagestyle{empty}
\begin{center}
\color{blue}\Large{Fernuniversität Hagen}\\
\end{center}


\begin{center}
%\Large{Bildung und Medien: eEducation}
\end{center}
\begin{verbatim}



\end{verbatim}
\begin{center}
\textbf{\Large{Kommentierte Bibliographie zum Thema Online-Lernen}}
\end{center}
\begin{verbatim}

\end{verbatim}
\begin{center}
%\textbf{im Studiengang Wirtschaftsinformatik}
\end{center}
\begin{verbatim}











\end{verbatim}

\begin{flushleft}
\begin{tabular}{lll}
\textbf{Studiengang:} & & MA Bildung und Medien: eEducation\\
& & Modul 2: Anwendungsbezogene Bildungsforschung\\
& & \\
& & \\
\textbf{eingereicht von:} & & {\color{magenta} Inge Koch-Meinass \flq{}ingekoch@mac.com\frq{}}\\
& & {\color{magenta}Matrikelnr.: 9650962 }\\
& & \\
\textbf{eingereicht am:} & & 09. November 2015\\
& & \\
& & \\
%\textbf{Betreuer:} & & Herr Prof. Dr. J. A. Müller
\end{tabular}
\end{flushleft}

% das ist wohl jetzt das Ende des Dokumentes
\end{titlepage}

%\listoftodos
\tableofcontents
\listoftables
\setcounter{page}{1}
%\thispagestyle{empty}
\pagebreak
\section{Einleitung}\label{einleitung}

Der ultimative Test Im folgenden werden 5 Artikel vorgestellt, die auf
unterschiedlichsten Ebenen den Erfolg von Online-Seminaren evaluieren.
Es stehen verschiedene Themengebiete als auch verschiedene Altersgruppen
im Mittelpunkt. Allen Artikeln gemein ist, dass der Lernerfolg mit
quantitativen Methoden ermittelt wird. Was genau unter Lernerfolg zu
verstehen ist, wird in den jeweiligen Studien definiert und hier
beschrieben.

\section{Ausgewählte Artikel}\label{ausgewuxe4hlte-artikel}

\textbf{\fullcite{hohenberg2009erfolgreiches}}

Gegenstand der Untersuchung ist eine dreijährige MTA-Ausbildung der
Fachrichtung Radiologie. Die Ausbildung geht über drei Jahre und wird
zeitgleich als Präsenzveranstaltung (Vollzeitausbildung) und als
Blended-Learning-Ansatz (Berufsbegleitend, alle 4-6 Wochen
Präsenzveranstaltung, ansonsten Online-Lerninhalte) angeboten.
Untersucht wurde der Zeitraum 2004 - 2007. Die \emph{Forschungsfrage}
lautet: Gibt es zwischen den Absolventen des Blended-Learning-Ansatzes
und den Absolventen der Präsenzveranstaltung unterschiedliche
Prüfungsleistungen? Die \emph{Stichprobe:} insgesamt haben an der
Weiterbildung 30 Personen teilgenommen, davon 20 Teilnehmer in der
Blended-Learning Gruppe und 10 Teilnehmer in der Präsenzgruppe. Die
Teilnehmer wurden nicht zufällig auf die Gruppen verteilt, sondern nach
Wunsch. In der Blended-Learning-Gruppe gab es einen männlichen
Teilnehmer, ansonsten waren alle Teilnehmer weiblich. Der Mittelwert des
Alters lag in der Blended-Learning-Gruppe bei 43 Jahren und in der
Präsenzgruppe bei 28 Jahren. Auch andere Parameter, wie Schulabschluss,
Familienstand, Kinder im Haushalt, Berufliche Bildung sind sehr
unausgewogen, was die Vergleichbarkeit der beiden Gruppen einschränkt.
Der wesentliche Unterschied zwischen den beiden Kursen bestand vor allem
darin, dass im Blended-Learning-Ansatzes die Module nacheinander und
nicht parallel abgearbeitet wurden. Lerninhalte waren in beiden
Lernsettings identisch. Nach jeder Einheit wurden Einsendeaufgaben
bearbeitet, die dann von einem Coach durchgesehen und kommentiert
wurden. Je nach Ergebnis wurde dann eventuell zusätzliche Lernmaterial
eingestellt. Das Ende der Ausbildung war für beide Gruppen eine
umfangreiche je gleiche Abschlussprüfung. Die schriftlichen, mündlichen
und praktischen Ergebnisse der Abschlussprüfungen wurden mittels des
Man-Whitney U-Testes auf signifikante Unterschiede überprüft. Ob
mögliche Unterschiede auch vom Alter der Teilnehmenden abhängig sind,
wird jeweils mittels eines Korrelationstestes überprüft. Die
\emph{Ergebnisse} sind je nach Fach sehr unterschiedlich: Die
schriftlichen Prüfungsfächer sind in zwei Themenblöcke,
naturwissenschaftliche Grundlagen und radiologische Spezialthemen,
unterteilt. In den naturwissenschaftlichen Grunglagenfächern gab es
insgesamt betrachtet keine signifikanten Unterschiede, wohl aber in
einigen Einzelfächer, in denen die Teilnehmenden des
Blended-Learning-Ansatzes signifikant besser abschnitten. Auch im
Fächerblock 2 erzielten die Teilnehmer des Blended-Learning-Ansatz
signifikant bessere Ergebnisse. Eine Abhängigkeit von Alter und
Lernerfolg bestand nicht. Bei den vier mündlichen Fächern, waren die
Ergenbnisse in einem Fach in der Blended-Learning-Gruppe signifikant
besser. Auch hier ergab der Korrelationstest keine Abhängigkeit von
Lernerfolg und Alter. Die praktischen Prüfungsergebnisse zeigten in zwei
Fächer signifikant bessere Ergebnisse zugunsten der Präsenzgruppe und in
einem Fach signifikant bessere Ergenbnisse zugunsten der
Blended-Learning-Gruppe. Die Korrelationsprüfung zeigte bei den
praktischen Prüfungen eine signifikante Abhängigkeit von
Prüfungsergebnis und Alter (je älter desto schlechter). Für die
\emph{Diskussion} der Ergebnisse sei vor allem hervorzuheben, das der
Vergleich der schriftlichen Prüfungsergebnisse keinerlei Einschränkungen
unterliegt, denn für alle Teilnehmer, unabhängig vom Lernsetting waren
die Prüfungen gleich. Das schlechtere Abschneiden der
Blended-Learning-Gruppe in den praktischen Fächern kann durch weniger
Übungsmöglichkeiten an modernen Geräten erklärt werden. Insgesamt
konnten die Teilnehmenden des Blended-Learning-Ansatzes mindestens
gleiche, teilweise sogar signifikant bessere Prüfungsergebnisse
erzielen. Die Ergebnisse sind aber aufgrund der geringen Teilnehmerzahl
und den heterogenen Gruppen auch kritisch zu hinterfragen. Zwar wurde
der Einfluss des Alters mit einbezogen, aber die unterschiedlichen
Schulabschlüsse Berufserfahrungsjahre usw. nicht. Die identischen
schriftlichen Prüfungen bieten aber eine statistisch gesehen sehr gute
Vergleichbarkeit und Ausgangsbasis für weiterführende Untersuchungen.

\textbf{\fullcite{lindemann2006multimedia}}

In dieser Untersuchung wurde der Online-Distance-Learning-Kurs VIRT.UM,
im Hinblick auf den Zusammenhang von Lernstiltyp, Multimedianutzung und
Lernerfolg evaluiert. VIRT.UM ist ein Online-Kurs der für Studierende
der Umweltwissenschaften an der Universität Zürich angeboten wird. Die
naturwissenschaftlichen Grundkenntnisse können hier online per
Selbststudium erworben werden. Alle Texte dieses Kurses sind sowohl
online als auch als Buch erhältich, während die Übungen, Fragen,
Multimediainhalte, Abbildungen usw. nur Online verfügbar sind. Die
Inhalte von VIRT.UM entsprechen zu 60\% denen des Buches und 40\% gehen
über die Inhalte der Lerntexte hinaus. Es wurden folgende
\emph{Forschungsfragen} gestellt:

\begin{enumerate}
\def\labelenumi{\arabic{enumi}.}
\itemsep1pt\parskip0pt\parsep0pt
\item
  Wählen Studierende selektiv unter den ihnen angebotenen
  Multimediatypen aus?
\item
  Falls ja hängt diese Auswahl vom jeweiligen Lernstiltyp ab?
\item
  Beeinflusst die Auswahl von Multimediatyp und der jeweilige
  Lernstiltyp den Lernerfolg der Studierenden?
\end{enumerate}

Das \emph{Versuchsdesign} kann wie folgt beschreiebn werden: Insgesamt
haben an der Evaluation 75 Studenten teilgenommen, sie wurden zufällig
in zwei gleich große Gruppen aufgeteilt. Zu Beginn wurde bei allen
Teilnehmern mittels eines Tests die Vorkenntnisse ermittelt. Eine Gruppe
erarbeite sich den Lernstoff online auf VIRT.UM, die andere Gruppe durch
eine herkömmliche Vorlesung. Der Wissenserwerb wurde dann mit einem Test
ermittelt. Danach wurden die Gruppen getauscht, nach dieser Phase wurde
ebenfalls der Wissenerwerb per Test ermittelt.

\textbf{\fullcite{Fischer2014a}}

Die vorliegende Studie ist Teil einer Dissertation, die sich mit
Entwicklung und Beforschung von interaktiven Lernmaterialien für
mathematische Brückenkurse befasst. Für bestimmte Studiengänge werden an
der Universität Kassel Vorkurse zur Mathematik angeboten. Die Kurse
werden von Studienanfängern unterschiedlichster Fachrichtungen besucht.
Ziel der Dissertation war einerseits die Entwicklung von interaktivem
Lernmaterial und andererseits die Evaluation der angebotenen Kurse. Das
Design der Studie basiert auf dem Angebots-Nutzungs-Modell der
Unterrichtswirksamkeit. Dieses Modell geht davon aus, dass Unterricht
als ein Angebot betrachtet wird, dass von den Lernenden genutzt werden
kann. In welcher Intensität dieses Angebot angenommen wird ist von
verschieden Faktoren (Lernpotenial, Kontext, Unterrichtsform usw.)
abhängig \parencite{Helmke2008}. Das Versuchsdesign sah wie folgt aus:
der mathematische Vorkurs konnte entweder als Präsenzveranstaltung
(P-Kurs) oder als Blended-Learning-Variante (E-Kurs) absolviert werden.
Es nahmen an der Studie etwa 1000 Studienanfänger teil. Entsprechend der
Wahl der Studenten für den E- oder P-Kurs wurden die zwei
Versuchsgruppen gebildet: von den 1000 Teilnehmern entschieden sich ca
290 Studierende für den E-Kurs. Für die im vorligenden Artikel
dargestellten Ergebnisse diente als Datenbasis ein elektronischer
Eingangs- und Ausgangstest in moodle. Am Eingangstest nahmen N=756
Teilnehmer statt, am Ausgangstest N=349. Außerdem fanden noch 3
Onlinebefragungen zu unterschiedlichen Zeitpunkten statt:
Eingangsbefragung mit N=586 Teilnehmer, Zwischentest mit N=400 und
Abschlussbefragung mit N=350. Untersucht wurde, ob die Ergebnisse der
Tests in Abhängigkeit von der Kursvariante signifikante Unterschiede
aufweisen. Die Untersuchung basiert auf dem Modell der
Unterrichtswirksamkeit von Helmke. Ausgewählte Ergebnisse dieser Studie
zeigte, dass die Wahl der Kursvariante hochsignifikanten abhängig war
von der Art der Hochschulzugangsberechtigung (allgemeines Abi oder
Fachabi), sowie von der Mathenote im Abitur. Um die Ergebnisse des
Abschlusstestes zu erklären wurde ein allgemeines lineares Modell
erstellt. Dies zeigte den oben beschriebenen starken Einfluss der
Ausgangsvorrausetzungen, aber auch einen schwach signifikanten Einfluss
der Kursvariante. D.h. Der Vergleich der Abschlusstestergebnisse von
Studenten mit gleichen Ergebnissen im Eingangstest ergab einen schwach
signifikanten Unterschied: Die Teilnehmer des E-Kurse hatten im Mittel
4,8 \% mehr Punkte. Die E-Kurs Teilnehmer hatten die Möglichkeit
elektronische Tests zu bearbeiten. Dies wurde zwar von Verhälnismäßig
wenigen Teilnehmern genutzt, dennoch zeigte sich eine signifikante
Abhängigkeit: Je mehr Tests von den jeweiligen Studenten bearbeitet
wurden, desto besser war das Abschneiden im Abschlusstest. Die E-Kurs
Teilnehmer, die alle Tests bearbeitet hatten, hatten in der
Abschlusspüfung im Mittel 14,4 \% mehr Punkte. (Verglichen wurden
Teilnehmer mit gleichen Eingagstestergebnissen)\\Weitere Ergebnisse
zeigen, dass es bezüglich der Nutzung der zur Verfügung gestellten
Lernmaterialen Unterschiede in Abhängigkeit der Studienrichtung gab. Im
Rahmen der hier durchgeführten Evaluation konnte dafür keine Erklärung
gefunden werden. Es treten bei dieser statistischen Untersuchung etliche
Störeinflüsse ein, die aber entsprechend behandelt werden, so dass die
Ergebnisse als valide angesehen werden können. Auch die Objektivität ist
gegeben, da der Eingangs- und Ausgangstest elektronisch durchgeführt
wurden und beispielsweise nicht von verschiedenen Prüfern abhängig ist.
Hervorzuheben ist auch die große Anzahl der Versuchsteilnehmer, was die
Aussagekraft der Ergebnisse unterstreicht. Die sehr differenzierte
Ausgestaltung der Tests auf der Grundlage von bewährten Methoden
(Unterrichtswirksamkeits-Modell) führt zu einer hohen Reproduzierbarkeit
der Studie.

\textbf{\fullcite{Nistor2005a}}

In dieser Studie wird die Akzeptanz, der Lernprozess und der Lernerfolg
problemorientierter virtueller Seminare mittels einer summativen
Wirkungsanalyse untersucht. Folgende zwei Seminare sind Gegenstand der
Wirkungsanalyse:

\begin{itemize}
\item
  ``Gestaltung und Evaluation virtueller Lernumgebungen'' Dieses
  Hauptseminar ist für Studierende aus den Fächern Pädagogik und
  Psychologie konzipiert. Ziel dieses Seminars Wissenerwerbs zur
  Gestaltung konstruktivistischer Lernumgebungen und die Evaluation
  virtueller Lernumgebungen. Außerdem soll ein Fragebogen entwickelt und
  eine Qualitätsanalyse durchgeführt werden. Das Seminar geht über 13
  Wochen und ist modular aufgebaut.
\item
  ``Einführung in das Wissensmanagement aus pädagogisch-psychologischer
  Sicht''
\end{itemize}

Dieses Hauptseminar ist konzipiert für Studenten der Fächer Pädagogik,
Psychologie, Informatik oder Betriebswirtschaftslehre. Die Studenten
sollen Grundlagen von Wissensmanagement kennen lernen, sowie einen
Wissensmanagement-Fall analysieren und systematisch bearbeiten können.
Das Seminar geht über 10 Wochen und ist ebenfalls modular aufgebaut.

Die Evaluation der Seminare erfolgte in den Dimensionen Akzeptanz,
Lernprozess und Lernerfolg.

Der Aufbau der virtuellen Seminare orientiert sich an einer
konstruktivistischen Auffassung von Lernen, nach welcher für die
Gestalung der Lernumgebung bestimmte Punkte erfüllt sein müssen:
Authenzität, Multiple Kontexte und Perspektiven, Soziale
Lernarrangements, Informations- Konstruktionsangebot, Instruktionale
Anleitung und Unterstützung

\textbf{\fullcite{mentzer2007two}}

\textbf{\fullcite{Sussman2010}}

\textbf{\fullcite{Crossley2015}}

\textbf{\fullcite{Edwards2013}}

\section{Fazit}\label{fazit}
\pagebreak
\printbibliography
\pagebreak
%\input{tabelleanalysen}
%\includepdf{alleine.pdf}
\end{document}
