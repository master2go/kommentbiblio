\documentclass[12pt, bibliography=totoc]{scrartcl}
\usepackage[headsepline,automark]{scrlayer-scrpage} %Trennlinie an Kopfzeile
%\usepackage{scrheadings}
\clearpairofpagestyles
\lohead{\rightmark}
\renewcommand{\partmark}[1]{\relax}% \part daran hindern, den Kolumnentitel zu löschen
\ohead[]{\pagemark}
%\ofoot*{\pagemark}
%%Kopfzeile
\usepackage{txfonts} %für times new roman
%\usepackage{helvet} %für arial, dann aber 11pt
\usepackage[a4paper, left=2.5cm, right=5cm]{geometry}
\usepackage[onehalfspacing]{setspace}
\usepackage{wasysym}
\usepackage{rotating}
%\usepackage{caption}
\usepackage[T1]{fontenc}
\usepackage[utf8]{inputenc}
%\usepackage{todonotes}
\usepackage{enumitem}
%\uespackage{caption}
\usepackage[bf]{caption}
\renewcommand{\captionfont}{\small\slshape}
\renewcommand{\figurename}{Abb.}
%\renewcommand{\thefigure}{\arabic{section}.\arabic{figure}}
%\makeatletter \@addtoreset{figure}{section} \makeatother
%\captionsetup[figure]{skip=1pt}
\usepackage{tabularx}
\usepackage{pdfpages}
\usepackage{array}
\usepackage{hyperref}
\usepackage{threeparttable} %fußnoten unterhalb tabelle
\usepackage{booktabs} % fuer schone Tabellen
\usepackage{rotating} % um tabellen auf quer drehen zu koennen http://www.golatex.de/kann-man-tabellen-im-querformat-darstellen-t2003.html
%\newcolumntype{C}[1]{>{\centering\arraybackslash}p{#1}} %Spalten mit fester breite zentriert
\newcolumntype{L}[1]{>{\raggedright\arraybackslash}p{#1}} %Spalten mit fester breite linksbündig
\newcolumntype{Y}{>{\small\raggedright\arraybackslash}X}
\newcolumntype{C}{>{\small\centering\arraybackslash}X}
\usepackage{graphicx}
%\usepackage[german]{babel}
\usepackage{typearea}
%\usepackage[style=apa]{biblatex}
\usepackage[style=apa,backend=biber]{biblatex}
\usepackage[american,ngerman]{babel}
\DeclareLanguageMapping{ngerman}{ngerman-apa}
\usepackage[babel,german=guillemets]{csquotes}

%nach part fängt section wieder mit eins an alte Gestaltung
%\makeatletter
%\@addtoreset{section}{part}
%\makeatother
\renewcommand*{\partformat}{\thepart}{}
\renewcommand*{\partheadmidvskip}{\nobreak\enskip}
%\bibliography{/Users/iNge/Dropbox/Biblio/ingesbibneu}
%\bibliography{/Users/iNge/Dropbox/Biblio/library}
\bibliography{library}
\begin{document}
%\begin{titlepage}
\thispagestyle{empty}
%\begin{center}
%\centering
\begin{minipage}[c]{1.1\textwidth}
%\textsc{\LARGE Fernuniversität von Hagen}\\[1.5cm]
%\textsc{\Large MA eEducation}\\[0.5cm]
\begin{center}

%\includegraphics[width=0.15\textwidth]{feulogo.jpg}\\[1cm]
{\Large {\color{blue} Kommentierte Bibliographie zum Thema Online-Lernen\\ [2cm]
{\large MA eEducation\\ Modul 2 Aufgabe 1 \\ WS 2015/16}\\[2cm]}
{\normalsize \bfseries \color{magenta}Inge Koch-Meinass\\ Alte Dorfstr. 99\\ 70599 Stuttgart\\ ingekoch@mac.com\\ Martrikelnummer: 965092}}


\end{center}
\end{minipage}
\end{titlepage}

\input{titelneu.tex}
%\listoftodos
\tableofcontents
\listoftables
\setcounter{page}{1}
%\thispagestyle{empty}
\pagebreak
\section{Einleitung}\label{einleitung}

Im folgenden werden 5 Artikel vorgestellt, die auf unterschiedlichsten
Ebenen den Erfolg von Online-Seminaren evaluieren. Es stehen
verschiedene Themengebiete als auch verschiedene Altersgruppen im
Mittelpunkt. Allen Artikeln gemein ist, dass der Lernerfolg mit
quantitativen Methoden ermittelt wird. Was genau unter Lernerfolg zu
verstehen ist, wird in den jeweiligen Studien definiert und hier
beschrieben.

\section{Ausgewählte Artikel}\label{ausgewuxe4hlte-artikel}

\textbf{\fullcite{hohenberg2009erfolgreiches}}

\fullcite{Crossley2015}

\parencite{Stocker2009}

\fullcite{Stocker2009}

Gegenstand der Untersuchung ist eine dreijährige MTA-Ausbildung der
Fachrichtung Radiologie. Die Ausbildung geht über drei Jahre und wird
zeitgleich als Präsenzveranstaltung (Vollzeitausbildung) und als
Blended-Learning-Ansatz (Berufsbegleitend, alle 4-6 Wochen
Präsenzveranstaltung, ansonsten Online-Lerninhalte) angeboten.
Untersucht wurde der Zeitraum 2004 - 2007. Die \emph{Forschungsfrage}
lautet: Gibt es zwischen den Absolventen des Blended-Learning-Ansatzes
und den Absolventen der Präsenzveranstaltung unterschiedliche
Prüfungsleistungen? Die \emph{Stichprobe:} insgesamt haben an der
Weiterbildung 30 Personen teilgenommen, davon 20 Teilnehmer in der
Blended-Learning Gruppe und 10 Teilnehmer in der Präsenzgruppe. Die
Teilnehmer wurden nicht zufällig auf die Gruppen verteilt, sondern nach
Wunsch. In der Blended-Learning-Gruppe gab es einen männlichen
Teilnehmer, ansonsten waren alle Teilnehmer weiblich. Der Mittelwert des
Alters lag in der Blended-Learning-Gruppe bei 43 Jahren und in der
Präsenzgruppe bei 28 Jahren. Auch andere Parameter, wie Schulabschluss,
Familienstand, Kinder im Haushalt, Berufliche Bildung sind sehr
unausgewogen, was die Vergleichbarkeit der beiden Gruppen einschränkt.
Der wesentliche Unterschied zwischen den beiden Kursen bestand vor allem
darin, dass im Blended-Learning-Ansatzes die Module nacheinander und
nicht parallel abgearbeitet wurden. Lerninhalte waren in beiden
Lernsettings identisch. Nach jeder Einheit wurden Einsendeaufgaben
bearbeitet, die dann von einem Coach durchgesehen und kommentiert
wurden. Je nach Ergebnis wurde dann eventuell zusätzliche Lernmaterial
eingestellt. Das Ende der Ausbildung war für beide Gruppen eine
umfangreiche je gleiche Abschlussprüfung. Die schriftlichen, mündlichen
und praktischen Ergebnisse der Abschlussprüfungen wurden mittels des
Man-Whitney U-Testes auf signifikante Unterschiede überprüft. Ob
mögliche Unterschiede auch vom Alter der Teilnehmenden abhängig sind,
wird jeweils mittels eines Korrelationstestes überprüft. Die
\emph{Ergebnisse} sind je nach Fach sehr unterschiedlich: Die
schriftlichen Prüfungsfächer sind in zwei Themenblöcke,
naturwissenschaftliche Grundlagen und radiologische Spezialthemen,
unterteilt. In den naturwissenschaftlichen Grunglagenfächern gab es
insgesamt betrachtet keine signifikanten Unterschiede, wohl aber in
einigen Einzelfächer, in denen die Teilnehmenden des
Blended-Learning-Ansatzes signifikant besser abschnitten. Auch im
Fächerblock 2 erzielten die Teilnehmer des Blended-Learning-Ansatz
signifikant bessere Ergebnisse. Eine Abhängigkeit von Alter und
Lernerfolg bestand nicht. Bei den vier mündlichen Fächern, waren die
Ergenbnisse in einem Fach in der Blended-Learning-Gruppe signifikant
besser. Auch hier ergab der Korrelationstest keine Abhängigkeit von
Lernerfolg und Alter. Die praktischen Prüfungsergebnisse zeigten in zwei
Fächer signifikant bessere Ergebnisse zugunsten der Präsenzgruppe und in
einem Fach signifikant bessere Ergenbnisse zugunsten der
Blended-Learning-Gruppe. Die Korrelationsprüfung zeigte bei den
praktischen Prüfungen eine signifikante Abhängigkeit von
Prüfungsergebnis und Alter (je älter desto schlechter). Für die
\emph{Diskussion} der Ergebnisse sei vor allem hervorzuheben, das der
Vergleich der schriftlichen Prüfungsergebnisse keinerlei Einschränkungen
unterliegt, denn für alle Teilnehmer, unabhängig vom Lernsetting waren
die Prüfungen gleich. Das schlechtere Abschneiden der
Blended-Learning-Gruppe in den praktischen Fächern kann durch weniger
Übungsmöglichkeiten an modernen Geräten erklärt werden. Insgesamt
konnten die Teilnehmenden des Blended-Learning-Ansatzes mindestens
gleiche, teilweise sogar signifikant bessere Prüfungsergebnisse
erzielen. Die Ergebnisse sind aber aufgrund der geringen Teilnehmerzahl
und den heterogenen Gruppen auch kritisch zu hinterfragen. Zwar wurde
der Einfluss des Alters mit einbezogen, aber die unterschiedlichen
Schulabschlüsse Berufserfahrungsjahre usw. nicht. Die identischen
schriftlichen Prüfungen bieten aber eine statistisch gesehen sehr gute
Vergleichbarkeit und Ausgangsbasis für weiterführende Untersuchungen.

\textbf{\fullcite{lindemann2006multimedia}}

In dieser Untersuchung wurde der Online-Distance-Learning-Kurs VIRT.UM,
im Hinblick auf den Zusammenhang von Lernstiltyp, Multimedianutzung und
Lernerfolg evaluiert. VIRT.UM ist ein Online-Kurs der für Studierende
der Umweltwissenschaften an der Universität Zürich angeboten wird. Die
naturwissenschaftlichen Grundkenntnisse können hier online per
Selbsstudium erworben werden. Alle Texte dieses Kurses sind sowol online
als auch als Buch erhältich, während die Übungen, Fragen,
Multimediainhalte, Abbildungen usw. nur Online verfügbar sind. Es wurden
folgende \emph{Forschungsfragen} gestellt:

\begin{enumerate}
\def\labelenumi{\arabic{enumi}.}
\itemsep1pt\parskip0pt\parsep0pt
\item
  Wählen Studierende selektiv unter den ihnen angebotenen
  Multimediatypen aus?
\item
  Falls ja hängt diese Auswahl vom jeweiligen Lernstiltyp ab?
\item
  Beeinflusst die Auswahl von Multimediaty und der jeweilige Lernstiltyp
  den Lernerfolg der Studierenden?
\end{enumerate}

\fullcite{mentzer2007two}

\fullcite{Sussman2010}

fullcite\{Edwards2013\}

\section{Fazit}\label{fazit}
\pagebreak
\printbibliography
\pagebreak
%\input{tabelleanalysen}
%\includepdf{alleine.pdf}
\end{document}
