% das Papierformat zuerst
%\documentclass[a4paper, 11pt]{article}

% deutsche Silbentrennung
%\usepackage[ngerman]{babel}
%\usepackage{color} \color{blue}
% wegen deutschen Umlauten
%\usepackage[utf8]{inputenc}

% hier beginnt das Dokument
%\begin{document}

\begin{titlepage}
\thispagestyle{empty}
\begin{center}
\color{blue}\Large{Fernuniversität Hagen}\\
\end{center}


\begin{center}
%\Large{Bildung und Medien: eEducation}
\end{center}
\begin{verbatim}



\end{verbatim}
\begin{center}
\textbf{\Large{Kommentierte Bibliographie zum Thema Online-Lernen}}
\end{center}
\begin{verbatim}

\end{verbatim}
\begin{center}
%\textbf{im Studiengang Wirtschaftsinformatik}
\end{center}
\begin{verbatim}











\end{verbatim}

\begin{flushleft}
\begin{tabular}{lll}
\textbf{Studiengang:} & & MA Bildung und Medien: eEducation\\
& & Modul 2: Anwendungsbezogene Bildungsforschung\\
& & \\
& & \\
\textbf{eingereicht von:} & & {\color{magenta} Inge Koch-Meinass \flq{}ingekoch@mac.com\frq{}}\\
& & {\color{magenta}Matrikelnr.: }\\
& & \\
\textbf{eingereicht am:} & & 09. November 2015\\
& & \\
& & \\
%\textbf{Betreuer:} & & Herr Prof. Dr. J. A. Müller
\end{tabular}
\end{flushleft}

% das ist wohl jetzt das Ende des Dokumentes
\end{titlepage}