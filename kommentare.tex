\section{Einleitung}\label{einleitung}

Im folgenden werden 5 Artikel vorgestellt, die auf unterschiedlichsten
Ebenen den Erfolg von Online-Seminaren evaluieren. Es stehen
verschiedene Themengebiete als auch verschiedene Altersgruppen im
Mittelpunkt. Allen Artikeln gemein ist, dass der Lernerfolg mit
quantitativen Methoden ermittelt wird. Was genau unter Lernerfolg zu
verstehen ist, wird in den jeweiligen Studien definiert. Es wurden die
Datenbanken Google Scholar, editlib.org und die deutschen Datenbanken
medienpaed.com nach entsprechenden Artikel durchsucht.

\section{Ausgewählte Artikel}\label{ausgewuxe4hlte-artikel}

\textbf{\fullcite{hohenberg2009erfolgreiches}}

Gegenstand der Untersuchung ist eine dreijährige MTA-Ausbildung der
Fachrichtung Radiologie. Die Ausbildung geht über drei Jahre und wird
zeitgleich als Präsenzveranstaltung (Vollzeitausbildung) und als
Blended-Learning-Ansatz (Berufsbegleitend, alle 4-6 Wochen
Präsenzveranstaltung, On-line-Lern-inhalte) angeboten. Untersucht wurde
der Zeitraum 2004 - 2007. Die \emph{Forschungsfrage} lautet: Gibt es
zwischen den Absolventen des Blended-Learning-Ansatzes und den
Absolventen der Präsenzveranstaltung unterschiedliche
Prüfungsleistungen? Die \emph{Stichprobe:} insgesamt haben an der
Weiterbildung 30 Personen teilgenommen, davon 20 Teilnehmer in der
Blended-Learning Gruppe und 10 Teilnehmer in der Präsenzgruppe. Die
Teilnehmer wurden nicht zufällig auf die Gruppen verteilt, sondern nach
Wunsch. In der Blended-Learning-Gruppe gab es einen männlichen
Teilnehmer, ansonsten waren alle Teilnehmer weiblich. Der Mittelwert des
Alters lag in der Blended-Learning-Gruppe bei 43 Jahren und in der
Präsenzgruppe bei 28 Jahren. Auch andere Parameter, wie Schulabschluss,
Familienstand, Kinder im Haushalt, Berufliche Bildung sind sehr
unausgewogen, was die Vergleichbarkeit der beiden Gruppen einschränkt.
Der wesentliche Unterschied zwischen den beiden Kursen bestand vor allem
darin, dass im Blended-Learning-Ansatzes die Module nacheinander und
nicht parallel abgearbeitet wurden. Lerninhalte waren in beiden
Lernsettings identisch. Nach jeder Einheit wurden Einsendeaufgaben
bearbeitet, die dann von einem Coach durchgesehen und kommentiert
wurden. Je nach Ergebnis wurde dann eventuell zusätzliche Lernmaterial
eingestellt. Das Ende der Ausbildung war für beide Gruppen eine
umfangreiche je gleiche Abschlussprüfung. Die schriftlichen, mündlichen
und praktischen Ergebnisse der Abschlussprüfungen wurden mittels des
Man-Whitney U-Testes auf signifikante Unterschiede überprüft. Ob
mögliche Unterschiede auch vom Alter der Teilnehmenden abhängig sind,
wird jeweils mittels eines Korrelationstestes überprüft. Die
\emph{Ergebnisse} sind je nach Fach sehr unterschiedlich: Die
schriftlichen Prüfungsfächer sind in zwei Themenblöcke,
naturwissenschaftliche Grundlagen und radiologische Spezialthemen,
unterteilt. In den naturwissenschaftlichen Grunglagenfächern gab es
insgesamt betrachtet keine signifikanten Unterschiede, wohl aber in
einigen Einzelfächer, in denen die Teilnehmenden des
Blended-Learning-Ansatzes signifikant besser abschnitten. Auch im
Fächerblock 2 erzielten die Teilnehmer des Blended-Learning-Ansatz
signifikant bessere Ergebnisse. Eine Abhängigkeit von Alter und
Lernerfolg bestand nicht. Bei den vier mündlichen Fächern, waren die
Ergenbnisse in einem Fach in der Blended-Learning-Gruppe signifikant
besser. Auch hier ergab der Korrelationstest keine Abhängigkeit von
Lernerfolg und Alter. Die praktischen Prüfungsergebnisse zeigten in zwei
Fächer signifikant bessere Ergebnisse zugunsten der Präsenzgruppe und in
einem Fach signifikant bessere Ergenbnisse zugunsten der
Blended-Learning-Gruppe. Die Korrelationsprüfung zeigte bei den
praktischen Prüfungen eine signifikante Abhängigkeit von
Prüfungsergebnis und Alter (je älter desto schlechter). Für die
\emph{Diskussion} der Ergebnisse sei vor allem hervorzuheben, das der
Vergleich der schriftlichen Prüfungsergebnisse keinerlei Einschränkungen
unterliegt, denn für alle Teilnehmer, unabhängig vom Lernsetting waren
die Prüfungen gleich. Das schlechtere Abschneiden der
Blended-Learning-Gruppe in den praktischen Fächern kann durch weniger
Übungsmöglichkeiten an modernen Geräten erklärt werden. Insgesamt
konnten die Teilnehmenden des Blended-Learning-Ansatzes mindestens
gleiche, teilweise sogar signifikant bessere Prüfungsergebnisse
erzielen. Die Ergebnisse sind aber aufgrund der geringen Teilnehmerzahl
und den heterogenen Gruppen auch kritisch zu hinterfragen. Zwar wurde
der Einfluss des Alters mit einbezogen, aber die unterschiedlichen
Schulabschlüsse Berufserfahrungsjahre usw. nicht. Die identischen
schriftlichen Prüfungen bieten aber eine statistisch gesehen sehr gute
Vergleichbarkeit und Ausgangsbasis für weiterführende Untersuchungen.

\textbf{\fullcite{lindemann2006multimedia}}

In dieser Untersuchung wurde der Online-Distance-Learning-Kurs VIRT.UM,
im Hinblick auf den Zusammenhang von Lernstiltyp, Multimedianutzung und
Lernerfolg evaluiert. VIRT.UM ist ein Online-Kurs der für Studierende
der Umweltwissenschaften an der Universität Zürich angeboten wird. Die
naturwissenschaftlichen Grundkenntnisse können hier online per
Selbststudium erworben werden. Alle Texte dieses Kurses sind sowohl
online als auch als Buch erhältich, während die Übungen, Fragen,
Multimediainhalte, Abbildungen usw. nur Online verfügbar sind. Die
Inhalte von VIRT.UM entsprechen zu 60\% denen des Buches und 40\% gehen
über die Inhalte der Lerntexte hinaus. Es wurden folgende
\emph{Forschungsfragen} gestellt:

\begin{enumerate}
\def\labelenumi{\arabic{enumi}.}
\itemsep1pt\parskip0pt\parsep0pt
\item
  Wählen Studierende selektiv unter den ihnen angebotenen
  Multimediatypen aus?
\item
  Falls ja hängt diese Auswahl vom jeweiligen Lernstiltyp ab?
\item
  Beeinflusst die Auswahl von Multimediatyp und der jeweilige
  Lernstiltyp den Lernerfolg der Studierenden?
\end{enumerate}

Das \emph{Versuchsdesign} kann wie folgt beschreiebn werden: Insgesamt
haben an der Evaluation 75 Studenten teilgenommen, sie wurden zufällig
in zwei gleich große Gruppen aufgeteilt. Zu Beginn wurde bei allen
Teilnehmern mittels eines Tests die Vorkenntnisse ermittelt. Eine Gruppe
erarbeite sich den Lernstoff online auf VIRT.UM, die andere Gruppe durch
eine herkömmliche Vorlesung. Der Wissenserwerb wurde dann mit einem Test
ermittelt. Danach wurden die Gruppen getauscht, nach dieser Phase wurde
ebenfalls der Wissenerwerb per Test ermittelt.

\textbf{\fullcite{Fischer2014a}}

Die vorliegende Studie ist Teil einer Dissertation, die sich mit
Entwicklung und Beforschung von interaktiven Lernmaterialien für
mathematische Brückenkurse befasst. Für bestimmte Studiengänge werden an
der Universität Kassel Vorkurse zur Mathematik angeboten. Die Kurse
werden von Studienanfängern unterschiedlichster Fachrichtungen besucht.
Ziel der Dissertation war einerseits die Entwicklung von interaktivem
Lernmaterial und andererseits die Evaluation der angebotenen Kurse. Das
Design der Studie basiert auf dem Angebots-Nutzungs-Modell der
Unterrichtswirksamkeit. Dieses Modell geht davon aus, dass Unterricht
als ein Angebot betrachtet wird, dass von den Lernenden genutzt werden
kann. In welcher Intensität dieses Angebot angenommen wird ist von
verschieden Faktoren (Lernpotenial, Kontext, Unterrichtsform usw.)
abhängig \parencite{Helmke2008}. Das Versuchsdesign sah wie folgt aus:
der mathematische Vorkurs konnte entweder als Präsenzveranstaltung
(P-Kurs) oder als Blended-Learning-Variante (E-Kurs) absolviert werden.
Es nahmen an der Studie etwa 1000 Studienanfänger teil. Entsprechend der
Wahl der Studenten für den E- oder P-Kurs wurden die zwei
Versuchsgruppen gebildet: von den 1000 Teilnehmern entschieden sich ca
290 Studierende für den E-Kurs. Für die im vorliegenden Artikel
dargestellten Ergebnisse diente als Datenbasis ein elektronischer
Eingangs- und Ausgangstest in moodle. Am Eingangstest nahmen N=756
Teilnehmer teil, am Ausgangstest N=349. Außerdem fanden noch 3
Onlinebefragungen zu unterschiedlichen Zeitpunkten statt:
Eingangsbefragung mit N=586 Teilnehmer, Zwischentest mit N=400 und
Abschlussbefragung mit N=350. Untersucht wurde, ob die Ergebnisse der
Tests in Abhängigkeit von der Kursvariante signifikante Unterschiede
aufweisen. Ausgewählte Ergebnisse dieser Studie zeigten, dass die Wahl
der Kursvariante hochsignifikant abhängig war von der Art der
Hochschulzugangsberechtigung (allgemeines Abi oder Fachabi), sowie von
der Mathenote im Abitur. Dies schränkt die Vergleichbarkeit von E- und
P-Kurs stark ein. Um die Ergebnisse des Abschlusstestes zu erklären
wurde ein allgemeines lineares Modell erstellt. Dies zeigte den oben
beschriebenen starken Einfluss der Ausgangsvorrausetzungen, aber auch
einen schwach signifikanten Einfluss der Kursvariante. D.h. Der
Vergleich der Abschlusstestergebnisse von Studenten mit gleichen
Ergebnissen im Eingangstest ergab einen schwach signifikanten
Unterschied: Die Teilnehmer des E-Kurse hatten im Mittel 4,8 \% mehr
Punkte. Die E-Kurs Teilnehmer hatten die Möglichkeit elektronische Tests
zu bearbeiten. Dies wurde zwar von vergleichsweise wenigen Teilnehmern
genutzt, dennoch zeigte sich eine signifikante Abhängigkeit: Je mehr
Tests von den jeweiligen Studenten bearbeitet wurden, desto besser war
das Abschneiden im Abschlusstest. Die E-Kurs Teilnehmer, die alle Tests
bearbeitet hatten, hatten in der Abschlusspüfung im Mittel 14,4 \% mehr
Punkte. (Verglichen wurden Teilnehmer mit gleichen
Eingagstestergebnissen)\\Weitere Ergebnisse zeigen, dass es bezüglich
der Nutzung der zur Verfügung gestellten Lernmaterialen Unterschiede in
Abhängigkeit der Studienrichtung gab. Im Rahmen der hier durchgeführten
Evaluation konnte dafür keine Erklärung gefunden werden.
\emph{Zusammenfassung:} Die Studie ist aufgrund der großen
Teilnehmerzahl sehr ausssagekräftig. Die zeitgleiche Durchführung von
E-Kurs und P-Kurs lässst unter Berücksichtigung, dass die Verteilung
nicht zufällig und starken Abhängigkeiten unterworfen war, eine gute
Vergleichbarkeit zu. Die Tests zu unterschiedlichen Zeitpunkten lassen
Aussagen auch über Teilnehmer zu, die vielleicht nicht bis zum Schluss
dabei bleiben. Einmal mehr zeigt sich die hohe Wirksamkeit von
Online-Seminaren und die damit verbundene Möglichkeit auch bei sehr
hohen Teilnehmerzahlen qualitativ hochwertige Lehre anzubieten.

\textbf{\fullcite{Nistor2005a}}

\emph{Versuchsbeschreibung}: In dieser Studie wird die Akzeptanz, der
Lernprozess und der Lernerfolg problemorientierter virtueller Seminare
mittels einer summativen Wirkungsanalyse untersucht. Folgende zwei
Seminare sind Gegenstand der Wirkungsanalyse:

\begin{itemize}
\item
  ``Gestaltung und Evaluation virtueller Lernumgebungen (Seminar eVAl)''
  Ziel dieses Seminars Wissenerwerbs zur Gestaltung konstruktivistischer
  Lernumgebungen und die Evaluation virtueller Lernumgebungen. Außerdem
  soll ein Fragebogen entwickelt und eine Qualitätsanalyse durchgeführt
  werden.
\item
  ``Einführung in das Wissensmanagement aus pädagogisch-psychologischer
  Sicht (Seminar Wissmann)'' Die Studenten sollen Grundlagen von
  Wissensmanagement kennen lernen, sowie einen Wissensmanagement-Fall
  analysieren und systematisch bearbeiten können.
\end{itemize}

Beide Seminare sind modular aufgebaut und gehen über mehrere Wochen.

Der Aufbau der virtuellen Seminare orientiert sich an einer
konstruktivistischen Auffassung von Lernen, nach welcher für die
Gestalung der Lernumgebung bestimmte Punkte erfüllt sein müssen:
Authenzität, Multiple Kontexte und Perspektiven, Soziale
Lernarrangements, Informations- Konstruktionsangebot, Instruktionale
Anleitung und Unterstützung. Die Evaluation der Seminare erfolgte in den
Dimensionen Akzeptanz, Lernprozess und Lernerfolg, woraus sich auch die
\emph{Forschungsfragen} ableiten:

\begin{enumerate}
\def\labelenumi{\arabic{enumi}.}
\itemsep1pt\parskip0pt\parsep0pt
\item
  Akzeptanz: Inwieweit akzeptieren die Lernenden die virtuellen
  Seminare?
\item
  Lernprozess: Wie beurteilen die Studierenden ihren Lernprozess in den
  virtuellen Seminaren?
\item
  Lernerfolg: Welche Lernergebnisse erzielen die Seminarteilnehmer?
\end{enumerate}

\emph{Methode}: Untersucht wurde im Somersemester 2004, das Seminar eVal
hatte eine Teilnehmerzahl von N=20, das Seminar Wissmann von N=26. Es
wurde einerseits mit schriftlichen Fragebögen evaluiert und um den
objektiven Wissenserwerb zu ermitteln mit Produktanalyse und
Wissenstest. Die eigens für diese Untersuchung entwickelten Fragebögen
umfassten insgesamt 58 geschlossene Items und, die auf einer Ratingskala
(1=trifft nicht zu bis 5=trifft vollkommen zu)beantwortet werden
konnten. Der Wissenstest enthielt in beiden Seminaren 6 inhaltliche
Fragen zu Fakten und Konzeptwissen. Die Ergebnisse wurden mit den
Antworten eines Experten verglichen und in \% davon ausgedrückt. Weiter
musste von den Seminarteilnehmern noch ein Testfall bearbeitet werden:
ein authentischer Problemfall sollte gelöst werden. Die Bewertung
entsprach der des Wissenstest. Fragebögen, Wissenstest und Testfall
wurden in der letzten Seminarwoche online ausgefüllt. \emph{Ergebnisse:}
Beide Seminare erfuhren eine hohe \textbf{Akzeptanz} (4.19 und 4.51),
bei gleichzeitiger Dropoutrate von 15\% bzw. 25\%. Beim
\textbf{Lernprozess} konnte eine sehr hohe Lernmotivation und eine
starke Kognitive Anregung durch die Problemorientierung ermittelt werden
(In beiden Seminaren jeweils über 4). Der Zusammnehang der beiden
Parameter war hochsignifikant. Der subjektive \textbf{Lernerfolg} lag
bei den verschiedenen Items in beiden Seminare mindstens bei 4.0.
Zwischen den Seminaren gas es keien Unterschiede, innerhalb des
Wissmannsseminars wurde das Faktenwissen signifikant höher eingeschätzt,
als das Anwendungswissen. Die objektive Einschaätzung des Lernerfolges
ergab bezüglich des Anwendungswissen sehr gute Bewertungen (82.69\% und
76.19\%), während die Ergebnisse von Fakten- und Anwendungswissen
deutlich darunter lagen.

\emph{Zusammenfassung}: Akzeptanz, Lernprozess und subjektiver
Lernerfolg wurden als hoch bis sehr hoch eingestuft und entsprichen den
Ergebnisse anderer Arbeiten zu virtuellen Seminaren. Auch der objektive
Lernerfolg, war teilweise sehr hoch, hier fehlt allerdings die
Vergleichbarkeit zu einer Kontrollgruppe (z.B.
Präsenzseminarteilnehmer). Insgesamt betrachtet zeigt diese Arbeit, wie
gut und erfogreich virtuelle Seminare in Verbindung mit dem Ansatz der
Problemorientierung angenommen werden und gibt sowohl Hinweise für die
Gestaltung, als auch detaillierte Anregungen für die Evaluation solcher
Seminare. Störfaktoren, wie die mangelnde Vergleichbarkeit, das Fehlen
der Aussagen der Abbrecher usw. werden ausführlich diskutiert und
bewertet.

\textbf{\fullcite{mentzer2007two}}

\emph{Inhalt:} Die Studie von \textcite{mentzer2007two} vergleicht
Lernergebnisse, Zufriedenheit und Art der Interaktion zwischen einem
Face-to Face (f2f) Lernsetting und einem Web-based-Kurs. Der Vergleich
findet statt bei Studenten der Studienrichtung Early Childhood.
\emph{Methode:} Die Kursteilnehmer werden zufällig auf zwei Gruppen
verteilt, eine Gruppe erarbeitet sich die Kursinhalte f2f, die andere
Gruppe online. Insgesamt haben 42 Studenten an dem Experiment
teilgenommen, N=24 in der Kontrollgruppe (f2f) und N=18 im
Web-Based-Course. Der Kurs der f2f findet an bestimmten Wochentagen
statt, während die Online-Gruppe immer und überall lernen kann. Die
Teilnehmer der Online-Gruppe müssen mind. zweimal pro Woche an
Live-Chats teilnehmen, um zu gewährleisten, dass sie mit dem Stoff
ebenso viel Zeit verbringen wie die f2f Teilnehmer. Zeitpaln und Inhalte
waren in beiden Gruppen gleich. Alle Teilnehmer führten einen Test durch
(VARK) mit dem der Lerntyp ermittelt wurde. Die Ausgewogenheit der
Gruppen bezüglich verschiedener Lerntypen wurde mit einem
Chi-Quadrat-Test überprüft, um statistisch signifikante Unterschiede in
der Teilnehmerstruktur ausschließen zu können. Um weitere Störungen
auszuschließen wurden beide Gruppen vom gleichen Lehrer unterrichtet,
die Ausgestaltung wurde dann nochmals von einem Kollegen auf Gleichheit
überprüft. Die Art der Interaktion wurde mit einem modifizierten
Analyseinstrument (IA) nach \textcite{Flanders1961} evaluiert. Dazu
wurden sowohl in der f2f Lernumgebung, als auch im Web-Based-Course die
Interaktionen beobachtet, kategorisiert und mengenmäßig erfasst. Für den
Vergleich wurden aus jedem Kurs zufällig 20 Minuten ausgewählt und
miteinander verglichen. \emph{Ergebnisse:} Die Interaktionsanalyse
zeigt, dass der Lehrer dazu tendiert im Web-Based-Course zeitlich
gesehen weniger zu unterrichten. Insgesamt zeigen die kurzen
ausgewerteten Abschnitte unterschiedliche Ergebnisse (mal zugunsten des
Web-Based-Course, mal zugunsten des f2f-Kurses). Ein wichtiges Ergebnis
war, dass im Chat Lehrer und Studenten eher gleichgestellt sind und sich
dadurch die Studenten nicht nur engagierter und mutiger beteiligen,
sondern auch eine wichtiger Rolle in der Diskussion einnehmen. Die
Zufriedenheit der Studenten mit dem Kursleiter, sowie mit dem Kurs im
allgemeinen war im f2f-Kurs signifikant größer als im Web-Based-Course
in beiden aber überdurchschnittlich gut. Um hier detaillierte
Informationen zu erhalten wurden die Effektgrößen der einzelnen Items
ermittelt.\\Die Prüfungsleistungen wurden mit eienr Zwischenprüfung und
einer Abschlussprüfung erhoben. Außerdem wurde die Gesamtnote des
Semester verglichen: signifikante Unterschiede gab es nur bei der
Gesamtnote: hier schnitten die Teilnehmer der f2f-Gruppe besser ab (A-),
als die des Web-Based-Kurses (B). Fazit: Die Untersuchung zeigt, dass es
in den verschiedenen Kursen, trotz gleicher Ausgangsbedingungen
unterschiedliche Erkenntnisse der Studenten gibt. Da sowohl Lehrer, als
auch die Durchmischung der Gruppe in beiden Gruppen gleich war muss
davon ausgegangen werden das das Lerndesign Einfluss auf die Aneignung
de Stoffes hat. Da jedoch für die Interaktionsanalyse nur kleine
Ausschnitte beobachtet wurden und die Teilnehmerzahl sehr klein und auf
eine Studienrichtung beschränkt war, lässt sich das Ergebnis nicht
verallgemeinern. Allerdings kann durch die zufällige Verteilung der
Studenten, davon ausgeangen werden, dass die Ergebnisse aus dem
Web-based-Course auf den durchschnittlichen Studenten zutreffen und
nicht nur auf besonders computeraffine Studenten. Kritisch ist
anzumerken, das es beim Online-Lernen ja eben nicht darum geht alles 1:1
zu übertragen, sondern auch die anderen und neuen Möglichkeiten des
Online-Lernens zu nutzen.

\textbf{\fullcite{Sussman2010}}

\textbf{\fullcite{Crossley2015}}

\textbf{\fullcite{Edwards2013}} \emph{Inhalt} 46 Schüler der 6. Klasse
nahmen im Fach Mathematik an einer Untersuchung teil, die klären sollte,
ob durch Online-Lernen oder Face-to Face lernen bessere Testergebnisse
erzielt werden können. Das \emph{Versuchsaufbau} entsprach einem
quasiexperimentellem, ausbalancierten Design mit Pre- und Posttest. D.h.
Zu einem gegebenen Zeitpunkt lernte eine Gruppe online eine andere im
Klassenzimmer und dann wurden die Gruppen getauscht. Es wurden 10
Themengebiete, in zwei Semestern auf diese Weise erarbeitet. Somit
erarbeitete jeder Schüler 5 Themngebiete online und 5 Themen
Face-to-Face. Zu Beginn jeden Semesters absolvierten alle Teilnehmer
einen Pretest, am Ende jeden Semesters einen Posttest. Um möglichst
gleiche Bedingungen zu gewährleisten, unterrichtete in beiden Lernformen
der gleichen Lehrer. Auch die Online-Lenphasen fanden im Klassenzimmer
statt, es gab keine Hausaufgaben. Unterschiedliche in der Literatur
beschriebenen Materialien, wie Videos, Spiele, Text usw. kamen zum
Einsatz. \emph{Ablauf} Der traditionelle Face-to-Face Unterricht begann
immer mit Input und Erklärung des Lehrers (Frontalunterricht), danach
wurden Fragen gestellt und Aufgabne bearbeitet. Der Lehrer unterstützte
wo notwendig. Die Online-Stunde began mit dem Holen des Laptops und dem
Aufsuchen der entsprechenden Website. Die Seite stellte den Stoff der
gesamten Einheit zur Verfügung, möglich war alles der Reihe nach
abzuarbeiten oder nach Wunsch. Keine Face-to-Face Kommunikation fand
statt, nur per online chat. Hier gestellte Fragen wurden sofort
beantwortet. Alle Lösungen wurden per email, durch die Website oder per
Chat eingereicht. \emph{Auswertung} Die Testergebnisse wurden mittels
eines t-Testes themenweise miteinander verglichen \emph{Ergebnisse} Bei
der Untersuchung des Lernzuwachses (Gain Scores) gab es in einem der 10
Topics einen signifikanten Unterschied. Dieser ist vermutlich auf einen
starken Unterschied im Pretest zurückzuführen und es ist demnach
fraglich, ob er reproduzierbar ist. Die Ergebnisse des Abschlusstestes
zeigen für keines der Themen einen signifikanten Unterschied. Auch alle
Ergebnisse zusammengenommen zeigten keine signifikanten Unterschiede.
Als weitere Differenzierung wurde noch nach dem TOST-Ansatz (two
one-sided t-tests) auf möglicherweise doch vorhandene Unterschiede
getestet. Dafür wurde durch den Lehrer angegeben bis zu welcher
Punktdifferenz ein Ergebnis noch als gleich angesehen werden kann. Die
Zone der Ungleichheit betrug demnach \pm{2.5} Punkte. Auch dieser Test
ergab keine signifikanten Unterschiede. \emph{Fazit} Es kann aus den
Ergebnissen gefolgert werden, dass im Hinblick auf die Testergebnisse
die Lernumgebung keine Rolle zu spielen scheint. Aus Sicht des Lehrers
gibt es für beide Settings Vor- und Nachteile. Vorteile Face-to-Face
Unterricht: Die Nähe zu den Schülern, Ermutigung, Motivation usw. ist in
diesem Kontext besser möglich. Für das Online-Lernen spricht: lernen im
eigenen Tempo, Vielfäligkeit der Methoden und bessere
Kommunikationsmöglichkeiten (direktes Fragen stellen ohne warten,
chatten untereinander) Die gleichen Testergebnisse und die
Ausgewogenheit von Vor- und Nachteilen zeigt an, dass eine Kombination
der Lehrmethoden in einem Blended-Learning-Ansatz eine sehr gute
Möglichkeit sein könnte die traditionellen Unterrichtsformen zu
erweitern.\\Die Ergebnisse sind aufgrund der speziellen Datenbasis nur
eingeschränkt übertragbar: ob diese in anderen Fächern, mit anderen
Teilnehmer aus anderen sozialen Schichten so vorzufinden sind bleibt
fraglich. Denoch zeigt sich das Online-Lernen auch bei jungen Lernern
sehr gut möglich ist, was zu der Frage führt wie jung können Lerner sein
um online lernen zu können?

\section{Fazit}\label{fazit}

Die Studien zeigen a. die Unterschiedlichkeit der Ergebnisse und b. die
Schwiergigkeit der Vergleiche. Wählen Studenten das lernsetting steckt
hierhin schon eine erhebliche Störung, gleichmäßig durchmischte
Versuchsgruppen sind nicht gegeben. Es fehlen Studien, bei denen die
Studierenden zufällig auf die VErsuchsgruppe verteilt werden. Die
Artikel machen deutlich, dass es bezühglich des ``Nutzens'' von
Online-Seminaren noch erheblichen Forschungsbedarf gibt. Die
Persönlichkeit und Lerntyp spielen eine große Rolle bei der Akzeptanz
der Lernangebote. Daher gilt auch für die Implementierung von
Online-Seminaren, dass
