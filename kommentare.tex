\section{Einleitung}\label{einleitung}

Im folgenden werden 5 Artikel vorgestellt, die auf unterschiedlichsten
Ebenen den Erfolg von Online-Seminaren evaluieren. Gegenstand der
Untersuchungen sind verschiedene Altersgruppen, als auch verschiedene
Lernformen (Weiterbildung, Studium, Schule). Allen Artikeln gemein ist,
dass der Lernerfolg mit quantitativen Methoden ermittelt wird. Was genau
unter Lernerfolg zu verstehen ist, wird in den jeweiligen Studien
definiert. Es wurden die Datenbanken Google Scholar, editlib.org und die
deutschen Datenbanken medienpaed.com nach entsprechenden Artikel
durchsucht. Die Treffer wurden dann nach Durchsicht der Abstracts und
Passung zur Aufgabenstellung ausgewählt. Es fand keine Sortierung
bezüglich der Ergebnisse, Themen oder Teilnehmenden statt.

\section{Artikel}\label{artikel}

\begin{enumerate}
\def\labelenumi{\arabic{enumi}.}
\item
  \fullcite{hohenberg2009erfolgreiches}
\item
  \fullcite{Fischer2014a}
\item
  \fullcite{Nistor2005a}
\item
  \fullcite{mentzer2007two}
\item
  \fullcite{Edwards2013}
\end{enumerate}

\pagebreak

\begin{enumerate}
\def\labelenumi{\arabic{enumi}.}
\itemsep1pt\parskip0pt\parsep0pt
\item
  \fullcite{hohenberg2009erfolgreiches}**
\end{enumerate}

Gegenstand der Untersuchung ist eine dreijährige MTA-Ausbildung der
Fachrichtung Radiologie. Die Ausbildung geht über drei Jahre und wird
zeitgleich als Präsenzveranstaltung (Vollzeitausbildung) und als
Blended-Learning-Ansatz (Berufsbegleitend, alle 4-6 Wochen
Präsenzveranstaltung, On-line-Lern-inhalte) angeboten. Untersucht wurde
der Zeitraum 2004 - 2007. Die \emph{Forschungsfrage} lautet: Gibt es
zwischen den Absolvent\_innen des Blended-Learning-Ansatzes und den
Absolvent\_innen der Präsenzveranstaltung unterschiedliche
Prüfungsleistungen? Die \emph{Stichprobe:} insgesamt haben an der
Weiterbildung 30 Personen teilgenommen, davon 20 in der Blended-Learning
Gruppe und 10 in der Präsenzgruppe. Die Auszubildenden wurden nicht
zufällig auf die Gruppen verteilt, sondern nach Wunsch. In der
Blended-Learning-Gruppe gab es bis auf eine Ausnahme nur weibliche
Teilnehmende. Der Mittelwert des Alters lag in der
Blended-Learning-Gruppe bei 43 Jahren und in der Präsenzgruppe bei 28
Jahren. Auch andere Parameter, wie Schulabschluss, Familienstand, Kinder
im Haushalt, Berufliche Bildung waren gruppenweise sehr unterschiedlich,
was die Vergleichbarkeit der beiden Gruppen einschränkt. Der wesentliche
Unterschied zwischen den beiden Kursen bestand vor allem darin, dass im
Blended-Learning-Ansatz die Module nacheinander und nicht parallel
abgearbeitet wurden. Lerninhalte waren in beiden Lernsettings identisch.
Nach jeder Einheit wurden Einsendeaufgaben bearbeitet, die dann von
einem Coach durchgesehen und kommentiert wurden. Je nach Ergebnis wurde
dann eventuell zusätzliche Lernmaterial eingestellt. Das Ende der
Ausbildung war für beide Gruppen eine umfangreiche je gleiche
Abschlussprüfung. Die schriftlichen, mündlichen und praktischen
Ergebnisse der Abschlussprüfungen wurden mittels des Man-Whitney
U-Testes auf signifikante Unterschiede überprüft. Ob mögliche
Unterschiede auch vom Alter der Teilnehmenden abhängig sind, wird
jeweils mittels einer Korrelationsanalyse überprüft. Die
\emph{Ergebnisse} waren je nach Fach sehr unterschiedlich: Die
schriftlichen Prüfungsfächer sind in zwei Themenblöcke,
naturwissenschaftliche Grundlagen und radiologische Spezialthemen,
unterteilt. In den naturwissenschaftlichen Grundlagenfächern gab es
insgesamt betrachtet keine signifikanten Unterschiede, wohl aber in
einigen Einzelfächer, in denen im Blended-Learning-Ansatzes signifikant
besser abschnitten wurde. Auch im Fächerblock 2 erzielten die
Teilnehmer\_innen des Blended-Learning-Ansatz signifikant bessere
Ergebnisse. Eine Abhängigkeit von Alter und Lernerfolg bestand nicht.
Bei den vier mündlichen Fächern, waren die Ergebnisse in einem Fach in
der Blended-Learning-Gruppe signifikant besser. Auch hier ergab die
Korrelationsanalyse keine Abhängigkeit von Lernerfolg und Alter. Die
praktischen Prüfungsergebnisse zeigten in zwei Fächern signifikant
bessere Ergebnisse zugunsten der Präsenzgruppe und in einem Fach
signifikant bessere Ergebnisse zugunsten der Blended-Learning-Gruppe.
Die Korrelationsanalyse zeigte bei den praktischen Prüfungen eine
signifikante Abhängigkeit von Prüfungsergebnis und Alter (je älter desto
schlechter). Für die \emph{Diskussion} der Ergebnisse sei vor allem
hervorzuheben, dass der Vergleich der schriftlichen Prüfungsergebnisse
keinerlei Einschränkungen unterliegt, denn für alle Teilnehmer\_innen,
unabhängig vom Lernsetting waren die Prüfungen gleich. Das schlechtere
Abschneiden der Blended-Learning-Gruppe in den praktischen Fächern kann
durch weniger Übungsmöglichkeiten an modernen Geräten erklärt werden.
Insgesamt konnten die Teilnehmenden des Blended-Learning-Ansatzes
mindestens gleiche, teilweise sogar signifikant bessere
Prüfungsergebnisse erzielen. Die Ergebnisse sind aber aufgrund der
geringen Teilnehmer\_innenzahl und den heterogenen Gruppen auch kritisch
zu hinterfragen. Zwar wurde der Einfluss des Alters mit einbezogen, aber
die unterschiedlichen Schulabschlüsse, Berufserfahrungsjahre usw. nicht.
Auch Vorkenntnisse wurden nicht ermittelt. Die identischen schriftlichen
Prüfungen und eine zufällige Verteilung auf die beiden Lernsettings,
wäre eine gute Möglichkeit für weitere Untersuchungen.

\begin{enumerate}
\def\labelenumi{\arabic{enumi}.}
\setcounter{enumi}{1}
\itemsep1pt\parskip0pt\parsep0pt
\item
  \fullcite{Fischer2014a}
\end{enumerate}

Die vorliegende Studie ist Teil einer Dissertation, die sich mit
Entwicklung und Beforschung von interaktiven Lernmaterialien für
mathematische Brückenkurse befasst. Für bestimmte Studiengänge werden an
der Universität Kassel Vorkurse zur Mathematik angeboten. Die Kurse
werden von Studienanfänger\_innen unterschiedlichster Fachrichtungen
besucht. Ziel der Dissertation war einerseits die Entwicklung von
interaktivem Lernmaterial und andererseits die Evaluation der
angebotenen Kurse. Das Design der Studie basiert auf dem
Angebots-Nutzungs-Modell der Unterrichtswirksamkeit. Dieses Modell geht
davon aus, dass Unterricht als ein Angebot betrachtet wird, dass von den
Lernenden genutzt werden kann. In welcher Intensität dieses Angebot
angenommen wird ist von verschieden Faktoren (Lernpotenial, Kontext,
Unterrichtsform usw.) abhängig \parencite{Helmke2008}. Das
\emph{Versuchsdesign} sah wie folgt aus: der mathematische Vorkurs
konnte entweder als Präsenzveranstaltung (P-Kurs) oder als
Blended-Learning-Variante (E-Kurs) absolviert werden. Es nahmen an der
Studie etwa 1000 Studienanfänger\_innen teil. Entsprechend der Wahl der
Studierenden für den E- oder P-Kurs wurden die zwei Versuchsgruppen
gebildet: von den 1000 Teilnehmer\_innen entschieden sich etwa 290
Studierende für den E-Kurs. Für die im vorliegenden Artikel
dargestellten Ergebnisse diente als Datenbasis ein elektronischer
Eingangs- und Ausgangstest in Moodle. Am Eingangstest nahmen N=756
Teilnehmer\_innen teil, am Ausgangstest N=349. Außerdem fanden noch 3
Onlinebefragungen zu unterschiedlichen Zeitpunkten statt:
Eingangsbefragung mit N=586 Teilnehmer\_innen, Zwischentest mit N=400
und Abschlussbefragung mit N=350. Untersucht wurde, ob die Ergebnisse
der Tests in Abhängigkeit von der Kursvariante signifikante Unterschiede
aufweisen. Ausgewählte Ergebnisse dieser Studie zeigten, dass die Wahl
der Kursvariante hochsignifikant abhängig war von der Art der
Hochschulzugangsberechtigung (allgemeines Abi oder Fachabi), sowie von
der Mathenote im Abitur. Dies schränkt die Vergleichbarkeit von E- und
P-Kurs stark ein. Um die Ergebnisse des Abschlusstestes zu erklären
wurde ein allgemeines lineares Modell erstellt. Dies zeigte den oben
beschriebenen starken Einfluss der Ausgangsvoraussetzungen, aber auch
einen Einfluss der Kursvariante: Der Vergleich der
Abschlusstestergebnisse von Studierenden mit gleichen Ergebnissen im
Eingangstest ergab einen schwach signifikanten Unterschied: Die
Teilnehmer\_innen des E-Kurses hatten im Mittel 4,8 \% mehr Punkte. Die
E-Kurs Teilnehmer\_innen hatten die Möglichkeit elektronische Tests zu
bearbeiten. Dies wurde zwar von vergleichsweise wenigen
Teilnehmer\_innen genutzt, dennoch zeigte sich eine signifikante
Abhängigkeit: Je mehr Tests von den jeweiligen Studierenden bearbeitet
wurden, desto besser war das Abschneiden im Abschlusstest. Die E-Kurs
Teilnehmer\_innen, die alle Tests bearbeitet hatten, hatten in der
Abschlussprüfung im Mittel 14,4 \% mehr Punkte. (Verglichen wurden
Teilnehmer\_innen mit gleichen Eingangstestergebnissen)\\Weitere
Ergebnisse zeigen, dass es bezüglich der Nutzung der zur Verfügung
gestellten Lernmaterialen Unterschiede in Abhängigkeit der
Studienrichtung gab. Im Rahmen der hier durchgeführten Evaluation konnte
dafür keine Erklärung gefunden werden. \emph{Zusammenfassung:} Die
Studie ist aufgrund der großen Teilnehmerzahl besonders aussagekräftig
und zeigt bei der E-Kurs Variante Vorteile im Wissenszuwachs. Die
zeitgleiche Durchführung von E-Kurs und P-Kurs lässt unter
Berücksichtigung, dass die Verteilung nicht zufällig und starken
Abhängigkeiten unterworfen war, eine gute Vergleichbarkeit zu. Die Tests
zu unterschiedlichen Zeitpunkten lassen Aussagen auch über
Teilnehmer\_innen zu, die vielleicht nicht bis zum Schluss dabei
bleiben. Einmal mehr zeigt sich die hohe Wirksamkeit von
Online-Seminaren und die damit verbundene Möglichkeit auch bei sehr
hohen Teilnehmer\_innenzahlen qualitativ hochwertige Lehre anzubieten.

\begin{enumerate}
\def\labelenumi{\arabic{enumi}.}
\setcounter{enumi}{2}
\itemsep1pt\parskip0pt\parsep0pt
\item
  \fullcite{Nistor2005a}
\end{enumerate}

\emph{Versuchsbeschreibung}: In dieser Studie wird die Akzeptanz, der
Lernprozess und der Lernerfolg problemorientierter virtueller Seminare
mittels einer summativen Wirkungsanalyse untersucht. Folgende zwei
Seminare sind Gegenstand der Wirkungsanalyse:

\begin{itemize}
\item
  ``Gestaltung und Evaluation virtueller Lernumgebungen (Seminar eVAl)''
  Ziel dieses Seminars ist der Wissenerwerbs zur Gestaltung
  konstruktivistischer Lernumgebungen und die Evaluation virtueller
  Lernumgebungen. Außerdem soll ein Fragebogen entwickelt und eine
  Qualitätsanalyse durchgeführt werden.
\item
  ``Einführung in das Wissensmanagement aus pädagogisch-psychologischer
  Sicht (Seminar Wissmann)'' Die Studierenden sollen Grundlagen von
  Wissensmanagement kennen lernen, sowie einen Wissensmanagement-Fall
  analysieren und systematisch bearbeiten können.
\end{itemize}

Beide Seminare sind modular aufgebaut und gehen über mehrere Wochen.

Der Aufbau der virtuellen Seminare orientiert sich an einer
konstruktivistischen Auffassung von Lernen, nach welcher für die
Gestaltung der Lernumgebung bestimmte Punkte erfüllt sein müssen:
Authenzität, Multiple Kontexte und Perspektiven, Soziale
Lernarrangements, Informations- Konstruktionsangebot, Instruktionale
Anleitung und Unterstützung. Die Evaluation der Seminare erfolgte in den
Dimensionen Akzeptanz, Lernprozess und Lernerfolg, woraus sich auch die
\emph{Forschungsfragen} ableiten:

\begin{enumerate}
\def\labelenumi{\arabic{enumi}.}
\itemsep1pt\parskip0pt\parsep0pt
\item
  Akzeptanz: Inwieweit akzeptieren die Lernenden die virtuellen
  Seminare?
\item
  Lernprozess: Wie beurteilen die Studierenden ihren Lernprozess in den
  virtuellen Seminaren?
\item
  Lernerfolg: Welche Lernergebnisse erzielen die
  Seminarteilnehmer\_innen?
\end{enumerate}

\emph{Methode}: Untersucht wurde im Sommersemester 2004, das Seminar
eVal hatte eine Teilnehmerzahl von N=20, das Seminar Wissmann von N=26.
Es wurde mit schriftlichen Fragebögen evaluiert und um den objektiven
Wissenserwerb zu ermitteln mit Produktanalyse und Wissenstest. Die
eigens für diese Untersuchung entwickelten Fragebögen umfassten
insgesamt 58 geschlossene Items, die auf einer Ratingskala (1=trifft
nicht zu bis 5=trifft vollkommen zu) beantwortet werden konnten. Der
Wissenstest enthielt in beiden Seminaren 6 inhaltliche Fragen zu Fakten
und Konzeptwissen. Die Ergebnisse wurden mit den Antworten eines
Experten verglichen und in \% davon ausgedrückt. Weiter musste von den
Seminarteilnehmer\_innen noch ein Testfall bearbeitet werden: ein
authentischer Problemfall sollte gelöst werden. Die Bewertung entsprach
der des Wissenstest. Fragebögen, Wissenstest und Testfall wurden in der
letzten Seminarwoche online ausgefüllt. \emph{Ergebnisse:} Beide
Seminare erfuhren eine hohe \textbf{Akzeptanz} (4.19 und 4.51), bei
gleichzeitiger Dropoutrate von 15\% bzw. 25\%. Beim \textbf{Lernprozess}
konnte eine sehr hohe Lernmotivation und eine starke kognitive Anregung
durch die Problemorientierung ermittelt werden (In beiden Seminaren
jeweils über 4). Der Zusammenhang der beiden Parameter war
hochsignifikant. Der subjektive \textbf{Lernerfolg} lag bei den
verschiedenen Items in beiden Seminare mindestens bei 4.0. Zwischen den
Seminaren gab es keine Unterschiede, innerhalb des Wissmannseminars
wurde das Faktenwissen signifikant höher eingeschätzt, als das
Anwendungswissen. Die objektive Einschätzung des Lernerfolges ergab
bezüglich des Anwendungswissen sehr gute Bewertungen (82.69\% und
76.19\%), während die Ergebnisse von Fakten- und Anwendungswissen
deutlich darunter lagen.

\emph{Zusammenfassung}: Akzeptanz, Lernprozess und subjektiver
Lernerfolg wurden als hoch bis sehr hoch eingestuft und entsprechen den
Ergebnisse anderer Arbeiten zu virtuellen Seminaren. Auch der objektive
Lernerfolg, war teilweise sehr hoch, insgesamt fehlt allerdings die
Vergleichbarkeit zu einer Kontrollgruppe (z.B.
Präsenzseminarteilnehmer\_innen). Dennoch zeigt diese Arbeit, wie gut
und erfolgreich virtuelle Seminare in Verbindung mit dem Ansatz der
Problemorientierung angenommen werden und gibt sowohl Hinweise für die
Gestaltung, als auch detaillierte Anregungen für die Evaluation solcher
Seminare. Störfaktoren, wie die mangelnde Vergleichbarkeit, das Fehlen
der Aussagen der Abbrecher usw. werden ausführlich diskutiert und
bewertet.

4.\fullcite{mentzer2007two}

\emph{Inhalt:} Die Studie von \textcite{mentzer2007two} vergleicht
Lernergebnisse, Zufriedenheit und Art der Interaktion zwischen einem
Face-to Face (f2f) Lernsetting und einem Web-based-Kurs. Die
Untersuchung wurde durchgeführt bei Studierenden der Studienrichtung
``Early Childhood''. \emph{Methode:} Die Kurseilnehmer\_innen wurden
zufällig auf zwei Gruppen verteilt, eine Gruppe erarbeitete sich die
Kursinhalte Face-to-Face, die andere Gruppe online. Insgesamt haben 42
Studierenden an dem Experiment teilgenommen, N=24 in der Kontrollgruppe
(f2f) und N=18 in der Experimentalgruppe (Web-Based-Kurs). Der Kurs der
f2f findet an bestimmten Wochentagen statt, während die Online-Gruppe
immer und überall lernen kann. Die Teilnehmer\_innen der Online-Gruppe
müssen mind. zweimal pro Woche an Live-Chats teilnehmen, um zu
gewährleisten, dass sie mit dem Stoff ebenso viel Zeit verbringen wie
die f2f Teilnehmer\_innen. Zeitplan und Inhalte waren in beiden Gruppen
gleich. Alle führten zu Beginn einen Test durch (VARK) mit dem der
Lerntyp ermittelt wurde. Die Ausgewogenheit der Gruppen bezüglich
verschiedener Lerntypen wurde mit einem Chi-Quadrat-Test überprüft, um
statistisch signifikante Unterschiede bezüglich der Lerngruppen
ausschließen zu können. Um weitere Störungen auszuschließen wurden beide
Gruppen vom gleichen Lehrer unterrichtet, die Ausgestaltung wurde dann
nochmals von einem Kollegen auf Gleichheit überprüft. Die Art der
Interaktion wurde mit einem modifizierten Analyseinstrument (IA) nach
\textcite{Flanders1961} evaluiert. Dazu wurden sowohl in der f2f
Lernumgebung, als auch im Web-Based-Kurs die Interaktionen beobachtet,
kategorisiert und mengenmäßig erfasst. Für den Vergleich wurden dann aus
jedem Kurs zufällig 20 Minuten ausgewählt. \emph{Ergebnisse:} Die
Interaktionsanalyse zeigt, dass der Lehrer dazu tendiert im
Web-Based-Kurs zeitlich gesehen weniger zu unterrichten. Insgesamt
zeigen die kurzen ausgewerteten Abschnitte unterschiedliche Ergebnisse
(mal zugunsten des Web-Based-Kurs, mal zugunsten des f2f-Kurses). Ein
wichtiges Ergebnis war, dass im Chat Lehrer und Studierenden eher
gleichgestellt sind und sich dadurch die Studierenden nicht nur
engagierter und mutiger beteiligen, sondern auch eine wichtiger Rolle in
der Diskussion einnehmen. Die Zufriedenheit der Studierenden mit dem
Kursleiter, sowie mit dem Kurs im allgemeinen war im f2f-Kurs
signifikant größer als im Web-Based-Kurs in beiden aber
überdurchschnittlich gut. Um hier detaillierte Informationen zu erhalten
wurden die Effektgrößen der einzelnen Items ermittelt. Die
Prüfungsleistungen wurden mit einer Zwischenprüfung und einer
Abschlussprüfung erhoben. Außerdem wurde die Gesamtnote des Semester
verglichen: signifikante Unterschiede gab es nur bei der Gesamtnote:
hier schnitten die Teilnehmer der f2f-Gruppe besser ab (A-), als die des
Web-Based-Kurses (B). Fazit: Die Untersuchung zeigt, dass die
Face-to-Face Studenten\_innen mit ihrem Kurs und dem Lehrer zufriedener
waren, während die Prüfungsergebnisse praktisch ohne Unterschiede waren.
Da der Unterrichtende jeweils gleich war, kann daraus geschlossen
werden, dass die unterschiedliche Akzeptanz auf das Lernsetting
zurückzuführen ist. Die Ergebnisse der Interaktionsanalyse können
aufgrund der sehr kleinen Datenbasis allenfalls Tendenzen aufzeigen.
Diese Studie ist besonders bemerkenswert weil sie eine der wenigen ist,
die eine zufällige Verteilung auf Face-to-Face bzw. Web-Based vornimmt.
Allerdings ist kritisch anzumerken, dass die 1:1 Übertragung der
Präsenzveranstaltung auf die Online-Veranstaltung, nicht repräsentativ
für Online- Seminare ist.

5.\fullcite{Edwards2013}

46 Schüler\_innen der 6. Klasse nahmen im Fach Mathematik an einer
Untersuchung teil, die klären sollte, ob durch Online-Lernen oder
Face-to Face lernen bessere Testergebnisse erzielt werden können. Das
\emph{Versuchsaufbau} entsprach einem quasiexperimentellem,
ausbalancierten Design mit Pre- und Posttest. D.h. Zu einem gegebenen
Zeitpunkt lernte eine Gruppe online eine andere Face-to-Face und dann
wurden die Gruppen getauscht. Es wurden 10 Themengebiete erarbeitet:
jede Teilnehmer\_in erarbeitete je 5 Themen online und 5 im
Präsenzunterricht. In jedem Semester wurden von alen Schüler\_innen Pre-
und Posttests absolviert. Um möglichst gleiche Bedingungen zu
gewährleisten, unterrichtete in beiden Lernformen der gleichen Lehrer.
Auch die Online-Lenphasen fanden im Klassenzimmer statt, es gab keine
Hausaufgaben. Unterschiedliche, in der Literatur beschriebene
Materialien wie Videos, Spiele, Text usw. kamen zum Einsatz.
\emph{Ablauf} Der traditionelle Face-to-Face Unterricht begann immer mit
Input und Erklärung des Lehrers (Frontalunterricht), danach wurden
Fragen gestellt und Aufgaben bearbeitet. Der Lehrer unterstützte wo
notwendig. Die Online-Stunde begann mit dem Holen des Laptops und dem
Aufsuchen der entsprechenden Website. Die Seite stellte den Stoff der
gesamten Einheit zur Verfügung, möglich war, alles der Reihe nach
abzuarbeiten oder nach Wunsch. Die Schüler\_innen konnten nur per online
chat miteinander kommunizieren. An den Leher gestellte Fragen im Chat
wurden sofort beantwortet. Alle Lösungen wurden per email, durch die
Website oder per Chat eingereicht. \emph{Auswertung} Die Testergebnisse
wurden mittels eines t-Testes themenweise miteinander verglichen.
\emph{Ergebnisse} Bei der Untersuchung des Lernzuwachses (Gain Scores)
gab es bei einem der 10 Themen einen signifikanten Unterschied. Dieser
ist vermutlich auf einen starken Unterschied im Pretest zurückzuführen
und es ist demnach fraglich, ob er reproduzierbar ist. Die Ergebnisse
des Abschlusstestes (Test Scores) zeigen für keines der Themen einen
signifikanten Unterschied. Auch alle Ergebnisse zusammengenommen zeigten
keine signifikanten Unterschiede. Als weitere Differenzierung wurde noch
nach dem TOST-Ansatz (two one-sided t-tests) auf möglicherweise doch
vorhandene Unterschiede getestet. Dafür wurde durch den Lehrer angegeben
bis zu welcher Punktdifferenz ein Ergebnis noch als gleich angesehen
werden kann. Die Zone der Ungleichheit betrug demnach \(\pm2.5\) Punkte.
Auch dieser Test ergab keine signifikanten Unterschiede. \emph{Fazit} Es
kann aus den Ergebnissen gefolgert werden, dass im Hinblick auf die
Testergebnisse die Lernumgebung keine Rolle zu spielen scheint. Aus
Sicht des Lehrers gibt es für beide Settings Vor- und Nachteile.
Vorteile Face-to-Face Unterricht: Die Nähe zu den Schüler\_innen,
Ermutigung, Motivation usw. ist in diesem Kontext besser möglich. Für
das Online-Lernen spricht: lernen im eigenen Tempo, Vielfältigkeit der
Methoden und bessere Kommunikationsmöglichkeiten (direktes Fragen
stellen ohne warten, chatten untereinander) Die gleichen Testergebnisse
und die Ausgewogenheit von Vor- und Nachteilen zeigt an, dass auch junge
Menschen, hier: Angehörige der Generation Z online Lernen können. Der
spezielle Versuchsaufbau und die detaillierten statistischen Tests
machen die Untersuchung sehr aussagekräftig. Eingeschränkt wird dies
allerdings durch eine kleine Datenbasis: ob diese in anderen Fächern,
mit anderen Teilnehmer\_innen aus anderen sozialen Schichten so
vorzufinden sind bleibt fraglich. Dennoch zeigt sich das Online-Lernen
auch bei jungen Lernern sehr gut möglich ist, was die Autoren zu der
spannenden Frage führt wie jung Lerner sein dürfen, um online lernen zu
können?

\section{Fazit}\label{fazit}

Es wurden in der vorliegenden Arbeit fünf Studien zum Thema
Online-Lernen bearbeitet. Vier Studien
(\cite{hohenberg2009erfolgreiches}, \cite{Fischer2014a},
\cite{mentzer2007two}, \cite{Edwards2013}) stellten direkte Vergleiche
zwischen online und Face-to-Face lernen an. Gemessen wurde der Erfolg an
Tests: in keiner der vier Untersuchungen führte das Online-Lernen zu
schlechteren Ergebnissen. Im Gegenteil: die Testergebnisse waren
mindestens gleich gut oder sogar besser. Selbst junge Lerner, lernten
ohne direkten Input durch einen Lehrer genauso gut, wie im
Frontalunterricht. In zwei Untersuchungen wurde auch die Akzeptanz der
unterschiedlichen Kurse ermittelt, sie war einmal sehr hoch
(\cite{Nistor2005a}) und einmal weniger hoch \cite{mentzer2007two} als
im Präsenzkurs. Die unterschiedlichen Ausgangslagen Teilnehmer, Fächer,
Versuchsdurchführungen und Auswertungen lassen nur schwer einen Schluss
zu, können aber zumindest tendenziell die hohe Wirksamkeit von
Online-Seminaren aufzeigen. Gleichzeitig wird die Schwierigkeit solcher
Vergleiche, aufgrund der gegebenen menschlichen ``Störfaktoren''
aufgezeigt und verdeutlicht, wie durchdacht Versuchsaufbau und
Auswertung sein müssen, um alle Lernparameter zu erfassen und fundierte
Aussagen treffen zu können.
